%
% File coling2014.tex
%
% Contact: jwagner@computing.dcu.ie
%%
%% Based on the style files for ACL-2014, which were, in turn,
%% Based on the style files for ACL-2013, which were, in turn,
%% Based on the style files for ACL-2012, which were, in turn,
%% based on the style files for ACL-2011, which were, in turn, 
%% based on the style files for ACL-2010, which were, in turn, 
%% based on the style files for ACL-IJCNLP-2009, which were, in turn,
%% based on the style files for EACL-2009 and IJCNLP-2008...

%% Based on the style files for EACL 2006 by 
%%e.agirre@ehu.es or Sergi.Balari@uab.es
%% and that of ACL 08 by Joakim Nivre and Noah Smith

\documentclass[11pt]{article}
\usepackage{coling2014}
\usepackage{times}
\usepackage{url}
\usepackage{latexsym}
\usepackage{todonotes}
%\usepackage[style=authoryear]{biblatex}

\definecolor{dark-cornflower-blue-2}{RGB}{17,85,204}
\definecolor{dark-green-2}{RGB}{56,118,29}
\definecolor{dark-blue}{RGB}{0,0,100}

\newcommand{\sk}[1]{{\color{dark-green-2} #1}}
\newcommand{\dr}[1]{{\color{dark-cornflower-blue-2} #1}}
\newenvironment{sketch}{\color{dark-green-2}}{\ignorespacesafterend}
\newenvironment{draft}{\color{dark-cornflower-blue-2}}{\ignorespacesafterend}

%\setlength\titlebox{5cm}

% You can expand the titlebox if you need extra space
% to show all the authors. Please do not make the titlebox
% smaller than 5cm (the original size); we will check this
% in the camera-ready version and ask you to change it back.


\title{AI Essay}

\author{First Author \\
  Affiliation / Address line 1 \\
  Affiliation / Address line 2 \\
  Affiliation / Address line 3 \\
  {\tt email@domain} \\\And
  Second Author \\
  Affiliation / Address line 1 \\
  Affiliation / Address line 2 \\
  Affiliation / Address line 3 \\
  {\tt email@domain} \\\And
  Third Author \\
  Affiliation / Address line 1 \\
  Affiliation / Address line 2 \\
  Affiliation / Address line 3 \\
  {\tt email@domain} \\}

\date{}

\begin{document}
\maketitle
\begin{abstract}
\sk{
Lorem ipsum dolor sit amet, consectetur adipiscing elit. Nullam molestie nibh ac erat vestibulum convallis. Nullam vel tellus at leo congue luctus. Mauris in ultrices massa, et efficitur nibh. Maecenas id erat vitae urna blandit consequat quis et risus. Sed nec luctus velit. Proin bibendum malesuada felis id accumsan. Quisque accumsan rhoncus dui, in hendrerit dui interdum id. Aenean felis felis, egestas a auctor at, lacinia varius nisi. Integer nec neque in augue semper dictum sit amet non mi. Pellentesque vehicula fermentum tortor, sit amet efficitur odio hendrerit a. Praesent pellentesque elementum ante, at scelerisque mi condimentum sit amet. Nullam sollicitudin, tortor et aliquet tincidunt, velit orci lacinia odio, eget posuere augue nisl vel ex. Nulla vitae dolor vulputate, laoreet mauris vitae, pretium purus. In enim lacus, maximus ac dignissim at, finibus ac erat. Etiam turpis metus, commodo non eros vitae, ultricies condimentum ligula. Nullam ante nisl, dignissim vitae nisl. 
}
\todo{placeholder abstract to see how much space it occupies}
\end{abstract}

\section{Introduction}
\label{sec:introduction}

It all started with the question "Can machines think?". We as human civilisation have been creating tools to solve problems since the beginning of our time. As we grew better at creating tools and collecting them together, we were able to build more and more complex machines upon each other. Then came the computer, a machine first designed as a computational tool, which we later gave the ability to improve itself. Inevitably, we started to feel the uncertainty of what this can lead to.

Alan Turing also \todo{also?} avoided answering this question by introducing another imitation game experiment \cite{turing1950computing}. Can machines behave indistinguishably as human from another human observer's perspective? This experiment itself raised another discussion among computer scientists and philosophers. As researchers in various areas improved from time of Turing, instead of asking if machines think, we would like to explore the question of whether machines can have minds.


% How to use colors %%%
\sk{This is sketchy text, ie Green.}

\dr{This is draft text, ie Blue}

\begin{sketch}
You can also write Green sketch like this...
\end{sketch}

\begin{draft}
...and draft text like this.
\end{draft}

%%%%%%%%%%%%%%

\subsection{Strong versus weak AI}

\begin{draft}
Before we dive into the discussion regarding to machines with their minds, a clear framework should be illustrated in order to achieve most fruitful discuss. We often hear "strong AI" and "weak AI" in domain of artificial intelligence research, but these phrases can actually have two different meanings. Since the overlapping on terms, one often accidentally associated with mixture of both concepts. We would like to clear the distinctions here: 

\subsubsection*{Strong / Weak AI used in Computational Theory of Mind}

John Searle is often regarded as the first person to use the term "strong AI" in this definition, in his "Minds, brains, and programs" published in 1980 \cite{searle1980minds}. According to John Searle, "strong AI" claims that programs (or programmed computers) which reacts correctly according to right input with right outputs, should be regarded as possessing a mind just like humans possessing minds. And "weak AI" claims this reaction is merely an imitation of mind, which cannot be accounted as real mind \todo{forward reference}. Searle's famous Chinese room thought experiment, which is discussed in more detail in Section \ref{sec:chinese_room}, serves as refutation to strong AI claims. In which he introduced "intentionality" is not possible to be included in any programs, or to say "intentionality" is not possible to be modelled. A man in a Chinese room can perform perfect task following the instructions and rules while he/she still knows absolutely nothing about Chinese, no intentionality is at all involved in tasks carried out. Another way of understanding the difference is by looking into the actual actions carried out in Chinese room experiment, one can describe the instruction-following guy only access the syntax rules of Chinese, not the semantics of Chinese he/she been given.

\subsubsection*{Strong / Weak AI used to describe the ability to solve general problems within a single agent}

Sometimes also referred to as full AI, strong AI in this branch can be defined as an agent which can demonstrate ability to perform any intellectual task human capable of \cite{kurzweil2005singularity}. That is an agent can solve not-predetermined tasks matching a human level. If there exists one artificial intelligence agent which can pass the Turing test, one can classify it as strong AI or full AI because it demonstrates the ability to perform tasks in terms of natural language dialogue equivalent (or indistinguishable from a third person's observation) to human level. Passing the Turing test might not suggest the agent can solve "any" task human can, which can be too vague too achieve (infinite list of task). However, general agreements on general intelligence must accomplish following:
\begin{enumerate}
\item reasoning
\item learning
\item planning
\item natural language communication
\item Knowledge representation (KR) 
\end{enumerate}

We will refer to the first definition of strong AI as Searle's idea on strong AI and the second as Turing's. Although discussion in either one of these branch often leads to some common arguments from another branch, it is better to clarify at beginning. 
\end{draft}

\subsection{What is a mind?}

\subsubsection*{Does simulated minds count as real mind?}

\dr{
As we mentioned earlier, Searle refuted the claim of simulation of mind itself should be counted as real mind. Till today this argument still remain topic of debate. So why do people waffle about whether a simulation of mind can be attributed as real mind or not? Another famous question of philosophy could be linked to this simulation problem, the mind-body problem. The 16th-century philosopher Rene Descartes is commonly acknowledged to be the first person who formalised this problem: "how the immaterial mind (or soul) could influence the material body?" and from where is the boundary of material crossing to immaterial world if one regard information flow as input and mind as processor of information. Like most philosophical problems, answer to mind-body problem seems to depend largely on one's personal believe.
}

\subsubsection*{Border between mindful creation and not?}

\todo{Good if everyone can contribute something here}

\begin{draft}
% Leif:

In order to better understand the concept of minds, we take a look at some known organisms. We assume that humans have minds since we are capable of conscious thought and abstract reasoning. Many animals are known to exhibit intelligent behaviour in both individuals and communities. For example, many species of fish, bird and some mammals are able to periodically travel great distances to the location of their birth once a year. Other species such as dogs, dolphins and ants are well known for their social behaviour.

Mackintosh noted that crows had the ability to carry information from experiment to experiment while pidgeons were notoriously bad at this, seemingly starting over from scratch \cite{mackintosh1988animals}. He refuted the then common idea that intelligence in animals could be ordered using the evolutionary tree, ie the "phylogentic hierarchy".

One can argue that humans are unique in being able to evolve as a species in a much faster rate than evolution. Cultural changes, advances in medicin and technology can all be considered a conscious evolution of the species. While stray dogs may not be able to construct metro-stations of their own, they can learn how to use them to navigate their neighbourhoods not so differently from their fellow human passengers \cite{moskowDogs}.

\end{draft}

% Article: Approaches to the study of animal intelligence, by Mackintosh 1988.

\subsubsection*{(Maybe a way to separate): Purpose-driven OR Meaning driven?}
 
 \begin{draft}
Our discussion regarding to strong/weak AI or Chinese room experiment started with ideas from video clip shown in class "Humans need not apply" \cite{grey2014humans}. Horses, once extensively used worldwide by human, are completely replaced by auto-mobiles. This analogy is used to compare current situation of human labours and near-future artificial intelligence replacing us. Could we become the next horses and eliminated from job market? Judging by technology today, the answer is almost certain \todo{certain in what way?}. But we humans are different from horses fundamentally in terms of roles on earth, this is why we have created civilisations and horses have not. We used horses for the purpose of transportation because they are good at going forward (provided with some incentives and/or punishments). If one sees this process from abstract point of view, the process of getting a horse to walk from Paris to Nantes is very much similar to process of getting our AI agent from Arad to Bucharest \todo{relating to the pathfinding example, or?}. A purpose-driven agent who only exists for serving limited collection of purposes (most commonly single purpose). Although in some occasions in our lives we read Shakespeare just to pass some final exams (very much like the horse previously mentioned \todo{I don't see the connection to horses here. /Björn}), most of the time we dive into deeper, more abstract processes of thinking for no direct causal result. A meaning driven agent who proceed carrying out tasks even no heuristics or feedbacks given. 

The difference of a meaning driven behaviour is clearly different from purpose driven ones, but its rarely found in nature with exception of human activities (and some specific activities of particular mammals like dolphin). Neocortex can be related to these high level consciousness but having neocortex doesn't implies meaning driven behaviours \todo{What is neocortex?}. 
\end{draft}

%%%%%%%%%%%%%%%%%%%%%%%%%%%%%%%%%%%%%%%%%%%%%%%%%%%%%%%%%%%%
\section{Discussion}
\label{sec:discussion}

\subsection{How to test if an agent possesses a mind?}

\begin{draft}
Goertzel notes that there are several tests for assessing whether an agent has achieved a human level of AGI \cite{goertzel2012agi}. The most classical would be the Turing test which would statistically test if the robot could be indistinguishable from a human \cite{turing1950computing}. A human judge would be ask several, possibly connected, questions over a text chat to both a robot and a human before attempting to guess which of the two participants really is human.

Another, more difficult, test would be to let a robot enroll in a university \cite{goertzel2012agi}. The robot would succeed if it was able to obtain a degree given the same conditions of passing as the human students.

Goertzel also listed some tests that could be assumed to require human intellect but could also be solved by a "narrow AI", i.e. specific to the task, \cite{goertzel2012agi}. At least one of the tests: "to be able to play an arbitrary video game based on experience only" was recently achieved by Google on the Atari game platform \cite{googleAtari}.

%Goertzel notes that there is little consensus among AI researchers as to which test would objectively determine whether a robot is at a human level of AGI \cite{goertzel2012agi}.
\end{draft}

\begin{sketch}
Since it is quite hard to know what a mind is in general we instead attempt to answer the question of whether a machine can possess a human mind, that is think like a human. To determine if machines can possess minds, one can go through some formulated tests:

\subsubsection*{Turing test}

Alan Turing proposed a game of imitation to test if a program can mimic human behaviour in his classical paper \cite{turing1950computing}. If a program can answer a set of written arbitrary questions that requires abstract reasoning and be mistaken for being human, Turing concludes such a program to be \_.

\subsubsection*{College test}

Another test was proposed by Goertzel [REF here]\todo{Which article? Tried to find, but unsure which. (Possible link in latex comment). /Björn}: if a machine could get a college degree, then it must be \_. \todo{Leif looks more into this}
% http://www.newscientist.com/article/mg21528813.600-what-counts-as-a-conscious-thinking-machine.html#.VUsv3RiYLKM
% http://link.springer.com/chapter/10.2991/978-94-91216-62-6_8
\end{sketch}

\subsubsection*{Nagel Bat argument}

\begin{draft}
The two tests above only test the behaviour of an agent, that is AGI, the "first" version of strong AI. When it comes to minds, consciousness and understanding others, Nagel made an interesting point \cite{nagel1974like}. He argues that a human can never understand what it is like to be a bat; we can only imagine what it would be for us to be a bat which is different from understanding what it is like for a bat to be a bat. A related example would be how does one’s native tongue (ex Swedish) sound like to a foreign person (ex Chinese), we can only imagine what it would be like because we only know what it sounds like to a Swedish person. If we accept Nagel's argument, one can draw the parallel to computers. Thus a bat expert might be able to predict and even imitate a bat while not being able to understand it internally, a computer could possibly imitate and predict human behaviour whilst not understanding them emotionally. For example, things like laughter, grief and art could be predicted but not understood.

\subsubsection*{Chinese Room Experiment}
\label{sec:chinese_room}

In any case, even if a machine would successfully understand all abstract parts of being human, it could be very hard to tell from just a smart imitating machine. Searle illustrated this in his famous thought experiment: the Chinese room \cite{searle1980minds} \todo{Is this reference correct?}. If an agent that speaks English natively but doesn't understand any Chinese is sealed away in a room together with a list of very thorough instructions of how to manipulate some input characters (presumably) written in Chinese, he might be able to produce some Chinese text but it doesn't mean he understand what he is doing. However, the people on the outside might not be able to distinguish that agent from a native Chinese speaker, assuming the instructions are sufficiently detailed and intelligently designed. Likewise, if a computer successfully could understand humans, we couldn't really tell the difference of that from a list of intelligent instructions executed by a thorough but non-understanding agent.

\end{draft}

\subsection{How does a simple mind arise?}
\begin{draft}

What might be considered as \textit{simple minds} can arise in, among others, animals and robotics. Of course, this depends on which definition of 'mind' that is used, a question which is definitely not trivial. On one hand, these 'minds' display complex and seemingly intelligent behaviours, but on the other hand we are able to describe and explain how these complex behaviours work, often using rather simple explanations. This falls into what is called \textit{complex systems}: complex behaviours arising from simple components. Examples of this, some of which are described below, are social insects and even single-celled organisms.

\subsubsection*{Slime mould}

Slime mould are unicellular amoeba which, when food is in short supply, can form into networks that coordinate and display behaviours similar to animals with simple brains. This is extraordinary, since individual cells are not known to have the ability to think. Yet somehow a large collection of these individuals is able to display a degree of intelligence when put together, something which can be seen in the example with slime mould. An experiment showed that a mass of slime mould was able to generate a rail network over Tokyo that was strikingly similar to the network actually used [REF]. \todo{7 may 11:56, Björn working here}
% http://scienceblogs.com/notrocketscience/2010/01/21/slime-mould-attacks-simulates-tokyo-rail-network/

\subsubsection*{Bee drones}

In robotics, a research team at Harvard is aiming to create microrobots with a size and behaviour similar to bees [REF]. The researchers hope that these robots will be able to efficiently collaborate as a "colony" in order to act as robotic pollinators.
% http://robobees.seas.harvard.edu/research

Another research team from the universities of Sheffield and Sussex have made progress in simulating parts of a bee's brain and uploading it to a UAV (drone), allowing it to fly and use simple navigation to avoid running into obstacles [REF].
% http://greenbrain.group.shef.ac.uk/research/

\end{draft}

%%%%%%%%%%%%%%%%%%%%%%%%%%%%%%%%%%%%%%%%%%%%%%%%%%%%%%%%%%%%
\section{Conclusion}
\label{sec:conclusion}

\subsection{Does it matter?}

\subsubsection*{Before we start, you have to know they don't care!}
\begin{draft}
Simulation of intelligence or real intelligence is in many sense not important issue for AI researchers working on delivering programs that finish designated tasks. As much as you wouldn't care if a screwdriver is actually a simulated solution or a real solution as long as it fixes your IKEA beds. AI researchers should therefore in absolute need as they produce more agents solving our problems in life, and some of which are practically unsolvable without these agents. We will return to this point when we discuss ethical perspectives of developing technology in later, but now leaving you a nice question to ponder, screwdriver is handy tool but in wrong hand could be weapons to harm others, did the inventor of screwdriver thought of this perspective and wouldn’t it be more safe to not release this potentially dangerous version of screwdriver and keep refining until he/she come up with a harmless version?
  
Well, since AI researchers are not so interested in debating mindfulness of machines, it shouldn't matter so much for discussion right? Why not consider following question: "If a computer, provided with correct software, can be considered as a creation with mind (and in some degree, consciousness), would it be crime of murder if I pull out its plug?"
\end{draft}

\begin{sketch}
\subsubsection*{Assuming a computer with proper software accounts as mindful creation, we wont be able to find out!}

\subsubsection*{Does physical differences impose different level of mutual understanding?}
\end{sketch}

\subsubsection*{Our limits need not apply}
\begin{draft}

if we regard programs  as form of intelligence and mindful thing, machine intelligence shouldn't be confined (or defined) by limitations of humans. Maybe one can argue computers will never be able to understand true semantics of our natural languages, but this could be very narrow view. Our natural languages are result of cultural evolution separated by geographical boundaries, and if we review history of mankind, the flaws in natural language causing misunderstanding or semantic mismatch costed us dearly. Yet all civilisation throughout the world seems to reach one common syntax when it comes to mathematics. The connection from its syntax to semantics are so absolute that we still use zero to represent very same concept when zero was invented. Computers and programs are born in world of math, binary or not, there exists no doubt that computers are much capable than humans in field of numbers. Needless to say the speed of computation carry out by computers, easily reaching to petaflop at the time of writing, can never be achieved by our biological limits. 

It is truth saying human brains are most complex masterpiece of nature, evolution made our brain perfect to handle any happening situation around us. We can see the world and filter out most critical information and process under amazing low power consumption. While machines still have huge difficulty identifying heels from sneakers, one has to consider the fundamental differences between human and machine. Analog information is easy task for us because we've survived in such analog world for thousands of years, but you can't perceive current temperature up to 3 decimal precision. 

Embracing such perspective, and one will find huge potential in development of machines. As mentioned, all our cultures are geologically separated, so are all the individuals of mankind. Our communication forms are still very limited, even with aid from machines. Given the power of almost instantaneous connection, evolution curve of machine must exceed ours. 
\end{draft}


%%%%%%%%%%%%%%%%%%%%%%%%%%%%%%%%%%%%%%%%%%%%%%%%%%%%%%%%%%%%
% include your own bib file like this:
\bibliographystyle{acl}
\bibliography{references}

\end{document}
